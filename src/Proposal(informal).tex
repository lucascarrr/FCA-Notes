\documentclass[11pt]{article}

% Packages for better typography and formatting
\usepackage[utf8]{inputenc}  % Handle UTF-8 encoding
\usepackage[T1]{fontenc}     % Better font encoding
\usepackage{lmodern}         % Improved font
\usepackage{microtype}       % Better spacing
\usepackage{geometry}        % Adjust margins
\usepackage{amsmath, amssymb} % Math symbols and formatting
\usepackage{graphicx}        % Include graphics
\usepackage{hyperref}        % Hyperlinks
\usepackage{enumitem}        % Customizable lists
\usepackage{xcolor}          % Color text
\usepackage{tcolorbox}       % Colored boxes
\usepackage{accents}
\usepackage{tikz}
\usepackage{caption}
\usepackage{multicol}
\usepackage{titlesec}
\usepackage[parfill]{parskip} % Use parskip package without extra space

%From M275 "Topology" at SJSU
\newcommand{\id}{\mathrm{id}}
\newcommand{\taking}[1]{\xrightarrow{#1}}
\newcommand{\inv}{^{-1}}

%From M170 "Introduction to Graph Theory" at SJSU
\DeclareMathOperator{\diam}{diam}
\DeclareMathOperator{\ord}{ord}
\newcommand{\defeq}{\overset{\mathrm{def}}{=}}

%From the USAMO .tex files
\newcommand{\ts}{\textsuperscript}
\newcommand{\dg}{^\circ}
\newcommand{\ii}{\item}

% % From Math 55 and Math 145 at Harvard
% \newenvironment{subproof}[1][Proof]{%
% \begin{proof}[#1] \renewcommand{\qedsymbol}{$\blacksquare$}}%
% {\end{proof}}

\newcommand{\liff}{\leftrightarrow}
\newcommand{\lthen}{\rightarrow}
\newcommand{\opname}{\operatorname}
\newcommand{\surjto}{\twoheadrightarrow}
\newcommand{\injto}{\hookrightarrow}
\newcommand{\On}{\mathrm{On}} % ordinals
\DeclareMathOperator{\img}{im} % Image
\DeclareMathOperator{\Img}{Im} % Image
\DeclareMathOperator{\coker}{coker} % Cokernel
\DeclareMathOperator{\Coker}{Coker} % Cokernel
\DeclareMathOperator{\Ker}{Ker} % Kernel
\DeclareMathOperator{\rank}{rank}
\DeclareMathOperator{\Spec}{Spec} % spectrum
\DeclareMathOperator{\Tr}{Tr} % trace
\DeclareMathOperator{\pr}{pr} % projection
\DeclareMathOperator{\ext}{ext} % extension
\DeclareMathOperator{\pred}{pred} % predecessor
\DeclareMathOperator{\dom}{dom} % domain
\DeclareMathOperator{\ran}{ran} % range
\DeclareMathOperator{\Hom}{Hom} % homomorphism
\DeclareMathOperator{\Mor}{Mor} % morphisms
\DeclareMathOperator{\End}{End} % endomorphism

\newcommand{\eps}{\epsilon}
\newcommand{\veps}{\varepsilon}
\newcommand{\ol}{\overline}
\newcommand{\ul}{\underline}
\newcommand{\wt}{\widetilde}
\newcommand{\wh}{\widehat}
\newcommand{\vocab}[1]{\textbf{\color{blue} #1}}
\providecommand{\half}{\frac{1}{2}}
\newcommand{\dang}{\measuredangle} %% Directed angle
\newcommand{\ray}[1]{\overrightarrow{#1}}
\newcommand{\seg}[1]{\overline{#1}}
\newcommand{\arc}[1]{\wideparen{#1}}
\DeclareMathOperator{\cis}{cis}
\DeclareMathOperator*{\lcm}{lcm}
\DeclareMathOperator*{\argmin}{arg min}
\DeclareMathOperator*{\argmax}{arg max}
\newcommand{\cycsum}{\sum_{\mathrm{cyc}}}
\newcommand{\symsum}{\sum_{\mathrm{sym}}}
\newcommand{\cycprod}{\prod_{\mathrm{cyc}}}
\newcommand{\symprod}{\prod_{\mathrm{sym}}}
\newcommand{\Qed}{\begin{flushright}\qed\end{flushright}}
\newcommand{\parinn}{\setlength{\parindent}{1cm}}
\newcommand{\parinf}{\setlength{\parindent}{0cm}}
% \newcommand{\norm}{\|\cdot\|}
\newcommand{\inorm}{\norm_{\infty}}
\newcommand{\opensets}{\{V_{\alpha}\}_{\alpha\in I}}
\newcommand{\oset}{V_{\alpha}}
\newcommand{\opset}[1]{V_{\alpha_{#1}}}
\newcommand{\lub}{\text{lub}}
\newcommand{\del}[2]{\frac{\partial #1}{\partial #2}}
\newcommand{\Del}[3]{\frac{\partial^{#1} #2}{\partial^{#1} #3}}
\newcommand{\deld}[2]{\dfrac{\partial #1}{\partial #2}}
\newcommand{\Deld}[3]{\dfrac{\partial^{#1} #2}{\partial^{#1} #3}}
\newcommand{\lm}{\lambda}
\newcommand{\uin}{\mathbin{\rotatebox[origin=c]{90}{$\in$}}}
\newcommand{\usubset}{\mathbin{\rotatebox[origin=c]{90}{$\subset$}}}
\newcommand{\lt}{\left}
\newcommand{\rt}{\right}
\newcommand{\bs}[1]{\boldsymbol{#1}}
\newcommand{\exs}{\exists}
\newcommand{\dps}[1]{\displaystyle{#1}}

\newcommand{\sol}{\setlength{\parindent}{0cm}\textbf{\textit{Solution:}}\setlength{\parindent}{1cm} }
\newcommand{\solve}[1]{\setlength{\parindent}{0cm}\textbf{\textit{Solution: }}\setlength{\parindent}{1cm}#1 \Qed}

\newcommand{\middleline}{
    \par\noindent\raisebox{.5\baselineskip}{\makebox[\linewidth]{\rule{0.5\textwidth}{0.4pt}}}\par
}

\newcommand{\calC}{\mathcal{C}}
\newcommand{\st}{\text{s.t.}}
\newcommand{\calD}{\mathcal{D}} 
\newcommand{\calB}{\mathcal{B}}
% Theorems 
% Define theorem-like environments
\newtheorem{lemma}{Lemma}[section]
\newtheorem{proposition}[lemma]{Proposition}
\newtheorem{definition}[lemma]{Definition}
% Geometry settings for better note-taking space
\geometry{
  a4paper,
  left=0.5in,
  right=0.5in,
  top=0.5in,
  bottom=0.5in,
}
\title{Towards a Notion of Defeasibility in Formal Concept Analysis}
\author{Lucas Carr}
\date{06.06.24}

\begin{document}
\maketitle

\section{Introduction}
\section{The Initial Case}
\subsection{A Defeasible Formal Context} 
We define a defeasible formal context a $\K := (G,M,I,<)$ where $G$ is a set of objects, $M$ a set of attributes, $I$ an incidence relation on $G \times M$ expressed as $gIm$ for $g\in G, m\in M$ (exactly as we see with typical formal contexts); and, $<$ a strict partial order over $G$. 

\subsection{The Derivation Operators}
Over the defeasible formal context $\K = (G,M,I,<)$ we introduce two operators; $(\underlinesymbol{X}^\circ)$ and $\DCl{X}$, $X\subseteq M$. Before we define these operators, we introduce a definition of minimality over objects in $G$. 

\begin{align}
  Min(A) := \{g \in A | \nexists h \in A, h < g\} 
\end{align}

We can now define the respective operators, $A \subseteq G, B\subseteq M$. 

\begin{align}
  \underlinesymbol{A}^\circ &:= \{m \in M | \forall g \in Min(A), gIm\} \\ 
  \underlinesymbol{B}^\circ &:= \{g \in G | \forall m \in B, \nexists h \in G h < g, gIm  \}
\end{align}

(1) Describes the minimally ranked objects in $G$. From here, (2) describes an operation from a set of objects $A \subseteq G$ to the attributes shared by the minimal elements in $A$. (3) Is an operation from a set of attributes $B\subseteq M$ which results in the \textit{minimal} set of objects which have all the attributes in $B$. 

We can apply these operators twice, $\DCl{B}$ where $A\subseteq M$; this procedure would give the set of attributes shared by the minimal objects which satisfy the properties $B$. Conversely, $\DCl{A}$ where $A\subseteq G$ describes the minimal set of objects whichs satisfy the properties satisfied by all the objects in $A$.  

For an example, consider the defeasible context $\K = \{G,M,I,<\}$, with $o_i < o_j$ if $i < j$ for all objects. 
\begin{figure}[h]
  \begin{center}
    \begin{tabular}{c|cccc}
              & $m_1$     & $m_2$     & $m_3$     & $m_4$     \\ \hline
        $o_1$ & $\times$  & $\times$  &           & $\times$  \\ 
        $o_2$ & $\times$  & $\times$  & $\times$  &           \\ 
        $o_3$ &           & $\times$  & $\times$  & $\times$  \\ 
        $o_4$ &           & $\times$  &           & $\times$ 
    \end{tabular}
  \end{center}
  \caption{Defeasible Formal Context}
  \label{Figure:Initial Defeasible Formal Context}
\end{figure}


Given $A = \{m_2, m_3\}$, we have that $\underlinesymbol{A}^\circ = \{o_1\}$, and $\DCl{A} = \{m_1, m_2, m_4\}$. Conversely, given $B = \{o_3, o_4\}$, $\underlinesymbol{B}^\circ = \{m_2, m_3, m_4\}$, $\DCl{B} = \{o_3\}$.\footnote[1]{I acknowledge that this second example could just be replaced with $Min(B)$, I will think more about this. } 

%-------------_%

\subsubsection{Properties of $(\underlinesymbol{\cdot}^\circ)^\circ$}

\textbf{nonmonotonic:} $A \subseteq B \not \Rightarrow \DCl{A} \subseteq \DCl{B}$

Assume $\DCl{\cdot}$ were monotonic; then $A\subseteq B \Rightarrow \DCl{A} \subseteq \DCl{B}$.

From Figure~\ref{Figure:Initial Defeasible Formal Context}, if we take the oo-operation on $A_1 = \{m_1, m_2\}$, we get $\DCl{A_2} = \{o_1\}^\circ = \{m_1, m_2, m_4\}$. Then, on $A_2 = \{m_1, m_2, m_3\}$, we have $\DCl{A_2} = \{o_2\}^\circ = \{m_1, m_2, m_3\}$. Obviously, we have $A_1 \subseteq A_2$ but $\DCl{A_1} \not \subset \DCl{A_2}$. Thus, the oo-operator is nonmonotonic. 

\textbf{extensive:} $X \subseteq \DCl{X}$

Assume that $\DCl{\cdot}$ is not extensive; then, there exists some $X \not \subseteq \DCl{X}$. Let $Y\subseteq G$ such that $Y = \underlinesymbol{X}^\circ$. Then $\underlinesymbol{Y}^\circ$ is the set of properties shared by all minimal objects in $Y$ (this minimal requirement is redundant since all elements of $Y$ are minimal by construction). Observe that $\underlinesymbol{Y}^\circ \equiv \DCl{X}$. Since $X \not \subseteq \DCl{X}$, there must be some $y\in Y$ such that $y^\circ \cap X \not = X$. That is, there is some object in $Y$ which does not have all the attributes from $X$. However, this is a contradiction, since $y$ would not be in $Y$ since it is not the case that $\forall m \in X, yIm$ - which is our definition of $(\underlinesymbol{\cdot}^\circ)$. 


\textbf{idempotent:} $\DCl{A} = \DCl{\DCl{A}}$

\end{document}
