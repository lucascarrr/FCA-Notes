\section{Implications}
\label{sec:Implications}

\subsection{Introduction}
If $M$ is a set of attributes, we say that $A \rightarrow B$ is an implication over $M$; $A,B \subseteq M$. A set $D\subseteq M$ respects an implication $A \rightarrow B$ if i) $A \not \subseteq D$ or ii) $B \subseteq D$ - the same as material implication. We can represent this notion as $D \models A \rightarrow B$. 

If we have a set of implications $L := \{I_1,\dots ,I_n\}$ we say that a subset $D \models L$ if it respects every implication in $L$: formally, $D \models L \iff \forall A \rightarrow B \in L, D \models A \rightarrow B$. 

\subsubsection{Implications in a Formal Context}
Suppose we have a formal context $\mathcal{K} := (G, M, I)$ and $A \rightarrow B$ where  $A, B \subseteq M$. We say that $K \models A \rightarrow B$ if $\forall g \in G, g' \models A \rightarrow B$. That is, a formal context is a model of some implication, if every object in $G$ has a intent which is a model of that implication. 

Given $K \models A \rightarrow B$, the following are equivalent: 
\[1. \forall g \in G, g' \models A\rightarrow B \]
\[2. B \subseteq A''\]
\[3. A' \subseteq B'\]

(1) is obvious. 

For (2), we need to show $K \models A \rightarrow B \iff B \subseteq A''$. 

$\Rightarrow$ We want to show that $\forall g \in G, g' \models A \rightarrow B \implies B \subseteq A''$ 
\begin{enumerate}
    \item Every object which has $A$ in its intent, also has $B$ in its intent. 
    \item This means, $\forall g \in G, A \subseteq g', B \subseteq g'$
    \item So $B$ is a subset of every object intent in $G$. 
    \item Thus, $B \subseteq A''$. 
\end{enumerate}

$\Leftarrow$ Show that $B \subseteq A'' \implies \forall g \in G, g' \models A \rightarrow B$. 
\begin{enumerate}
    \item $A'''$ is the set of objects which have $A$ in their intents
    \item We know that all of these objects will have $B$ in their intents too (assumption)
    \item So, $\forall g \in G$ if $g\in A'''$ then $B \in g'$ and so $g' \models A \rightarrow B$. 
    \item $\forall g \in G$ and $g \not \in A'''$, then $g' \models A \rightarrow B$ trivially. 
    \item $\forall g \in G, g' \models A \rightarrow B$
\end{enumerate}

[We still need to do proof for (3)] 
\clearpage

\subsection{Sets of Implications}
Let $L$ be a set of implications over $M$, then 
\[Mod(L) := \{T \in \mathcal{P}(M) | T \models L\}\]
$Mod(L)$ is the set of all subsets of $M$ which are models of the set of implications $L$. This is a closure system on $M$. We can see that $M \in Mod(L)$ because for any implication $A \rightarrow B$ over $M$, $M \models A\rightarrow B$. 
\begin{proof}
    $M \models A\rightarrow B$ for any implication over $M$ with $A, B \subseteq M$. 
    \begin{enumerate}
        \item Let $A\rightarrow B$ be some implication over $M$. 
        \item Suppose $M \not \models A \rightarrow B$. 
        \item Then $A \subseteq M$ and $B \not \subseteq M$ 
        \item Contradiction, since $A, B \subseteq M$
    \end{enumerate}
\end{proof}
\begin{proof}
    The intersection of any elements in $Mod(L)$ is also in $Mod(L)$. 
    \begin{enumerate}
        \item Let $X, Y \in Mod(L)$, 
        \item Assume $X \cap Y \not \in Mod(L)$, 
        \item Then, for some $A\rightarrow B \in L, X \cap Y \not \models A \rightarrow B$, 
        \item $X \cap Y \subseteq A$ and $X \cap Y \not \subseteq B$. 
        \item But $Mod(L)$ is such that $\forall m \in Mod(L), A \subseteq m \implies B \subseteq m$. 
        \item If $X \cap Y \subseteq A$ then $X \cap Y \subseteq B$. 
        \item Thus, $X \cap Y \in Mod(L)$
    \end{enumerate}
\end{proof}

The above proves that $Mod(L)$ is a closure system on $M$. (From the definition in \ref{sec:Closure_Systems}) If L is the set of all implications of a formal context, then $Mod(L)$ is the system of all concept intents. 

Since $Mod(L)$ is a closure system, we have a closure operator $X \mapsto L(X)$, which can be described in two ways. 

$X^\mathcal{L} := X \cup \bigcup\{B | A \rightarrow B \in \mathcal{L}, A \subseteq X\}$. 

We then keep applying $X^\mathcal{L}$ until we get a fixed point $\mathcal{L}(X) := X^{\mathcal{L}\dots \mathcal{L}}$ with $\mathcal{L}(X)^\mathcal{L} = \mathcal{L}(X)$. 

Alternatively, we define the closure operator as $\mathcal{L}(X) := \bigcap\{Y | X \subseteq Y \subseteq M, Y \in Mod(L)\}$ 

These two definitions are equivalent; the first one is an iterative definition, which says: given a subset of $M$, $X$ - keep applying all the implications from $\mathcal{L}$ until we reach a fixed point. The second definition says that we define the closure of $X$ as the intersection between all the models of $\mathcal{L}$ which contain $X$. 
\clearpage

\subsection{When does an implication follow from other ones?}
Let $\mathcal{L}$ be a set of implications, and $A\rightarrow B$ an implication over $M$. How do we know if $\mathcal{L} \models A\rightarrow B$. Or, how do we know if an implication follows semantically from a set of implications? If every subset of $\mathcal{L}$ which respects $\mathcal{L}$ also respects $A\rightarrow B$. 

\begin{proof}
    $\mathcal{L} \models A\rightarrow B \iff B \subseteq \mathcal{L}(A)$

    $\mathcal{L} \models A\rightarrow B \Rightarrow B \subseteq \mathcal{L}(A)$
    \begin{enumerate}
        \item Assume $B \not \subseteq \mathcal{L}(A)$
        \item Then, $\exists Y \in Mod(\mathcal{L})$ \st $A \subset Y$ and $B \not \subseteq Y$.
        \item $Y \not \models A \rightarrow B$ and $Y \in Mod(\mathcal{L})$
        \item Contradiction, since $\mathcal{L} \models A\rightarrow B$
    \end{enumerate}

    $B \subseteq \mathcal{L}(A) \Rightarrow \mathcal{L} \models A\rightarrow B$ by contrapositive.
    \begin{enumerate}
        \item Assume $\mathcal{L} \not \models A \rightarrow B$ 
        \item Then, there exists some element $I \in Mod(\mathcal{L})$ with $A\subseteq I$ and $B \not \subseteq I$. 
        \item By inspection, $I \in \{Y | A \subseteq Y \subseteq M, Y \in Mod(\mathcal{L})\}$
        \item Which would then mean that $B \not \subseteq \bigcap \{Y | A \subseteq Y \subseteq M, Y \in Mod(\mathcal{L})\}$ 
    \end{enumerate}
\end{proof}