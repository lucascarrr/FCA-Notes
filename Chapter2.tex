\section{Closure Systems}
\label{sec:Closure_Systems}

\subsection{Defining a Closure System}
\label{subsec:Closure_Systems-Defining_a_Closure_System}

If we are given a set $M$, a closure system $\calC$ on $M$ is a set of subsets of $M$ ($\calC \subseteq \mathcal{P}(M)$) for which: 

\begin{itemize}
    \item $M \in \calC$
    \item if $D\subseteq \calC$ then, $\bigcap D \in \calC$
\end{itemize}

We might observe that the first point is ensured by the second point; this is because $\bigcap \emptyset = M$ - if we don't have this property, we have a contradiction. We might also observe that a closure system has a natural ordering; specifically, by subsumption. That is, $\calC$ is a poset with $(\calC, \subseteq)$. 

\subsection{Closure Operators}
\label{subsec:Closure_Systems-Closure_Operators}
